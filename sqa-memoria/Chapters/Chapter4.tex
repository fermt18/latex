% Chapter 4

\chapter{Test Automation} % Main chapter title

\label{Chapter4} % For referencing the chapter elsewhere, use \ref{Chapter1} 

%----------------------------------------------------------------------------------------

% Define some commands to keep the formatting separated from the content 
%\newcommand{\keyword}[1]{\textbf{#1}}
%\newcommand{\tabhead}[1]{\textbf{#1}}
%\newcommand{\code}[1]{\texttt{#1}}
%\newcommand{\file}[1]{\texttt{\bfseries#1}}
%\newcommand{\option}[1]{\texttt{\itshape#1}}

%----------------------------------------------------------------------------------------

\section{Overview}
\paragraph This section explains each part of the Jenkins based pipeline for application building and test automation. As pipelines were an important part in the SQA course, the focus here is more in the jobs that deal with test environment management and test execution. However, a simple job for building the application as well as its dependencies is also implemented.

\paragraph Main difference with the pipelines seen in the course is the use of Docker for deploying the services in the desired test environment. Through a set of configuration files located in the source control repository the services are deployed and the network for connecting each other depending on the test case is configured. The commands that trigger the deployment are executed by the Test Framework before running any set of test cases that need a specific deployment.

\paragraph So besides the mere reason of learning the tool, the use of Docker responds to the necessity of managing test environments automatically so the continuous integration chain is not broken even when running at different test levels.

%----------------------------------------------------------------------------------------

\section{Pipeline}
\paragraph The pipeline consists on 2 jobs:

\begin{itemize}
\item \textit{Build:} compiles the application from a clean Jenkins workspace with Maven and generates the JAR file. It also creates the Docker image from the generated JAR file. Although there are several services, only one JAR is generated and each independent service is run depending on the arguments (check \ref{Chapter2} for reference).

\item \textit{Test:} triggers the Cucumber Runner class targeting the folder where the feature files are. Each feature preconditions will make sure that the correct environment is up and running and each feature post conditions will clean the environment again. So the entire set of features including integration and system levels should be executed without stopping at any time.
\end{itemize}


%----------------------------------------------------------------------------------------

\section{Environment Deployment Approach}
\paragraph The features described in chapter \ref{Chapter3} can be split in three main groups:

\begin{itemize}
\item Integration Level - functional
\item System Level - functional
\item System Level - non functional
\end{itemize}

\paragraph The features within the Integration group would only need three services running at the same time as a prerequisite (Web service, Registration service and the service under test) while the features within the System Level group would need all the services to be up and running as the System Level wants to test end to end test cases and besides to look for unknown dependencies or collisions between the services deployed even if they are independent from each other.

